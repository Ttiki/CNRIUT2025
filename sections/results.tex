Lors de cette première année, nous avons mis en œuvre une méthodologie structurée pour collecter les indicateurs et les besoins spécifiques des différentes tâches de l’alliance UNITA. À partir d'entretiens collaboratifs, nous avons récupéré une liste d’indicateurs clés, validés en partenariat avec chaque équipe.

Ce processus est en cours de finalisation grâce à une deuxième série d’interviews dédiée à la validation des données. Parmi les 150 indicateurs d’outputs et 75 indicateurs d’outcomes identifiés, chacun doit être validé par les parties prenantes avant son intégration complète dans le data warehouse (DWH).

Une fois les données nettoyées pour éliminer incohérences et doublons, elles seront intégrées dans le DWH centralisé de l’alliance UNITA. Ce système facilite le suivi des progrès, l’identification des tendances, et la prise de décisions stratégiques. Lorsqu’un indicateur est validé pour une tâche, le DWH est configuré pour permettre la collecte des données auprès des institutions membres. Cela amorce les travaux d’analyse, notamment la création de rapports et la prédiction de l’impact.

La saisie et l’intégration des données se font de manière semi-automatique via des formulaires standardisés. Les institutions partenaires remplissent ces formulaires pour transmettre les données de leurs universités, qui sont ensuite enregistrées directement dans le DWH. Ces informations sont transformées pour alimenter UNITApedia, une solution MediaWiki sémantique connectée au DWH. Cet outil fournit des rapports interactifs et un accès transparent aux données pour tous les acteurs impliqués.

La Table~\ref{tab:indic_ex} illustre une sélection d’indicateurs pertinents utilisés pour mesurer les résultats (outputs) et les effets (outcomes) des activités de l’alliance UNITA.

\begin{table}[h]
    \caption{Exemples d’indicateurs outputs (sélection)}
    \centering
    \begin{tabular}{|l|l|}
        \hline
        \textbf{Indicateurs} & \textbf{Unités} \\ \hline
        Pourcentage de projets UNITA évalués & \% \\ \hline
        Nombre d’événements de matchmaking & Nombre \\ \hline
        Nombre de publications scientifiques & Nombre \\ \hline
    \end{tabular}
    \label{tab:indic_ex}
\end{table}
