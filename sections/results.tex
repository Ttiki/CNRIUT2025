Pour cette première année, nous avons établi une méthodologie pour récupérer les indicateurs et besoins de chaque tâche dans l'alliance UNITA. De ces interviews utilisant cette méthodologie orientée impact, nous avons pu récupérer une liste conséquente d'informations. Informations que nous avons validées avec chaque tâche. Enfin, après un premier nettoyage des données, elles furent utilisées au sein du data warehouse de UNITA en tant que méta-données.
\begin{table}[h]
  \caption{Résultat des premières interviews}
  \begin{center}
    \begin{tabular}{|l|l|} \hline
    \textbf{Type d'indicateurs} & \textbf{Nombre d'indicateurs} \\ \hline
    Outputs & 160 \\ \hline
    Outcomes & 75 \\ \hline
    \end{tabular}
  \end{center}
\end{table}

\vspace{-16px}