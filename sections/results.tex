Pour cette première année, nous avons établi une méthodologie pour récupérer les indicateurs et besoins de chaque tâche dans l'alliance UNITA. De ces interviews utilisant cette méthodologie orientée impact, nous avons pu récupérer une liste conséquente d'informations. Informations que nous avons validées avec chaque tâche. Enfin, après un premier nettoyage des données, elles furent utilisées au sein du data warehouse de UNITA en tant que méta-données. Pour l'instant, nous nous retrouvons avec environ 150 indicateurs \textit{d'outputs} et 75 \textit{d'outcomes}

\begin{table}[h] \caption{Exemples d'indicateurs (sélection)} \begin{center} \begin{tabular}{|l|l|} \hline \textbf{Indicateur} & \textbf{Unité} \\ \hline Pourcentage de projets UNITA évalués & \% \\ \hline Nombre d'événements de matchmaking & Nombre \\ \hline Nombre de publications scientifiques & Nombre \\ \hline Nombre de participants aux événements & Nombre \\ \hline Nombre de followers sur les réseaux sociaux & Nombre \\ \hline \end{tabular} \end{center} \end{table}

\vspace{-16px}