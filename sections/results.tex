Lors de cette première année, nous avons mis en œuvre une méthodologie structurée pour collecter les indicateurs et besoins spécifiques des différentes tâches de l’alliance UNITA. À partir d'entretiens collaboratifs, une liste d’indicateurs clés a été établie et validée avec chaque équipe.

Ce processus se finalise grâce à une seconde série d’interviews dédiée à la validation des données.

Une fois les données nettoyées pour éliminer incohérences et doublons, elles seront intégrées dans le DWH centralisé. Ce système facilite le suivi des progrès, l’identification des tendances et la prise de décisions stratégiques. Une fois validé, chaque indicateur peut être collecté auprès des institutions membres pour amorcer les travaux d’analyse : création de rapports et prédiction de l’impact.

La saisie et l’intégration des données se font semi-automatiquement via des formulaires standardisés remplis par les institutions partenaires, puis enregistrés dans le DWH. Ces données alimentent UNITApedia, une solution MediaWiki sémantique connectée au DWH, offrant des rapports interactifs et un accès transparent aux données pour tous les acteurs.

La Table~\ref{tab:indic_ex} illustre une sélection d’indicateurs pertinents utilisés pour mesurer les résultats (outputs) et les effets (outcomes) des activités de l’alliance UNITA.
\vspace{-16px}
\begin{table}[h]
    \caption{Exemples d’indicateurs outputs (sélection)}
    \centering
    \begin{tabular}{|l|l|}
        \hline
        \textbf{Indicateurs} & \textbf{Unités} \\ \hline
        Pourcentage de projets UNITA évalués & \% \\ \hline
        Nombre d’événements de matchmaking & Nombre \\ \hline
        Nombre de publications scientifiques & Nombre \\ \hline
    \end{tabular}
    \label{tab:indic_ex}
\end{table}
\vspace{-22px}