Pour cette première année, nous avons établi une méthodologie pour récupérer les indicateurs et besoins de chaque tâche dans l'alliance UNITA. De ces interviews utilisant cette méthodologie orientée impact, nous avons pu récupérer une liste conséquente d'informations. Informations que nous avons validées avec chaque tâche.
Un nettoyage des données a été nécessaire pour éliminer les erreurs et les incohérences. Les données nettoyées ont été intégrées dans le data warehouse (DWH) de UNITA en tant que méta-données. Le DWH stocke et gère les données de l'alliance UNITA, permettant de suivre les progrès et les résultats, identifier les tendances et les opportunités, et prendre des décisions éclairées.
Nous avons ainsi obtenu environ 150 indicateurs d'outputs et 75 indicateurs d'outcomes, qui nous permettent de mesurer les résultats et les impacts de l'alliance UNITA. Le tableau \ref{tab:indic_ex} montre une sélection parmi ces indicateurs.
\vspace{-12px}
\begin{table}[h] \caption{Exemples d'indicateurs (sélection)} \begin{center} \begin{tabular}{|l|l|} \hline \textbf{Indicateurs} & \textbf{Unités} \\ \hline Pourcentage de projets UNITA évalués & \% \\ \hline Nombre d'événements de matchmaking & Nombre \\ \hline Nombre de publications scientifiques & Nombre \\ \hline \end{tabular} \label{tab:indic_ex} \end{center} \end{table}

\vspace{-26px}