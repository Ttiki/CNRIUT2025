Les alliances universitaires ont attiré l'attention, notamment depuis le lancement en 2019 de l'Initiative des universités européennes (IUE). À mesure que les alliances s'étendent, il est nécessaire d'évaluer leur impact sociétal et de développer des stratégies de suivi. Cet examen décrit la recherche sur la mesure de l'impact, la Théorie du Changement (TdC) et les systèmes de Suivi et d'Évaluation (S\&E) au sein de méta-organisations comme les alliances universitaires. En explorant ces sujets, nous visons à construire un cadre pour évaluer leur impact. Nous définirons d'abord les concepts clés de notre étude.

\subsection{Impact}
L'impact varie selon la définition donnée par chaque organisation, mais se réfère généralement aux effets à long terme des activités sur leur environnement. L'impact dans cette recherche utilise la définition du CAD de l'OCDE : 

"Effets à long terme positifs et négatifs, primaires et secondaires, produits par une intervention de développement, directement ou indirectement, intentionnels ou non" \cite{oecd_quality_2010}.

Identifier et évaluer les indicateurs est un défi en raison de la difficulté à recueillir des preuves. Pour simplifier, les résultats des actions peuvent être segmentés en parties, chacune menant à la suivante dans une chaîne d'impact, comme illustré dans la Figure \ref{fig:simplified-impact-chain} \cite{stein_understanding_2012}.
\begin{figure} [h]
    \centering 
    \includegraphics[width=1\linewidth]{images/Diagrams-IMPACT.png} 
    \caption{Chaîne d'impact simplifiée\cite{peersman_when_2016}}
    \label{fig:simplified-impact-chain} 
\end{figure}

\begin{itemize} 
    \item Entrées : Représentent les ressources et matériaux nécessaires pour initier un projet ou un programme. 
    \item Activités : Actions et tâches spécifiques accomplies pour transformer les entrées en résultats significatifs. 
    \item Produits : Résultats directs et immédiats des activités, souvent quantifiables et mesurables. 
    \item Résultats : Indiquent les réalisations initiales issues des produits et des activités, contribuant aux objectifs du projet. 
    \item Impact : Représente les effets à long terme et les changements plus larges qui se produisent à cause du projet, influençant les parties prenantes et la communauté. 
\end{itemize}

Initialement larges, les méthodologies pour l'évaluation de l'impact sociétal ont évolué, notamment pour les alliances universitaires européennes durables

\subsection{Théorie du changement}
La théorie du changement dans les organisations a évolué au fil des ans, en commençant par la "Théorie des trois étapes du changement" de Kurt Lewin (1951). Lippitt, Watson et Westley ont élargi cela à sept étapes en 1958; Prochaska et DiClemente ont introduit un modèle cyclique pour le changement de comportement en matière de santé. La Théorie de l’action raisonnée et du comportement planifié indique que le comportement est motivé par des intentions. Elle introduit le "contrôle perçu", suggérant que les individus doivent sentir qu'ils ont les ressources, les opportunités et les compétences nécessaires pour effectuer le changement. Ces théories se croisent à divers niveaux mais manquent d'une théorie sociétale du changement. Nous visons à explorer la TDC de la Littératie Politique, axée sur la compréhension du développement du changement avec une approche politiquement informée, essentielle pour la planification stratégique, le suivi, l'évaluation, et la communication avec les partenaires sur les changements à venir.

\subsection{Système de Suivi et d'Évaluation }
Un système de Suivi et d'Évaluation (S\&E) se compose de deux composants : l'évaluation, qui quantifie régulièrement les mesures, et le suivi, qui détecte en continu les anomalies en utilisant les résultats de l'évaluation. Les questions clés pour les systèmes S\&E incluent quand et comment les mettre en œuvre au cours du cycle de vie d'un projet, s'il est possible de le faire après le début du développement, et s'ils sont nécessaires. Ignorer ces questions peut conduire à des systèmes inefficaces et à de mauvaises données. Malgré la demande pour la connaissance des systèmes S\&E, la plupart des recherches manquent de conseils détaillés sur la mise en œuvre et ne fournissent pas de cadre flexible adaptable à diverses utilisations.

\subsection{Data warehouse et base de connaissances partagée}
Notre objectif est d'explorer des solutions d'entrepôt de données basées sur la connaissance. Une grande partie de l'information est virtuelle et non quantifiable. Bien que nous puissions obtenir des données mesurables de nos partenaires, nous manquons d'aperçus partagés à travers la connaissance de l'alliance. L'impact, un concept abstrait, est souvent évalué à travers des indicateurs statistiques qui reflètent des résultats plutôt que l'impact lui-même. Prédire l'impact à travers ces indicateurs est courant, mais comprendre la base de connaissances d'une alliance est essentiel pour une évaluation précise. 