La globalisation et l'internationalisation de l'éducation ont connu une accélération spectaculaire ces dernières années, en particulier au sein des établissements d'enseignement supérieur. Ce mouvement mondial s'est popularisé en Europe grâce à l'initiative Erasmus+ des universités européennes. En 2019, l'initiative des universités européennes (EUI) a été lancée avec pour mission de promouvoir, de faciliter et de diffuser la nécessité d'alliances interuniversitaires en Europe. Depuis, 64 alliances, couvrant plus de 35 pays, regroupant plus de 500 universités ont été créées. Ces alliances entraînent des changements et des transformations importantes, apportant de nouveaux besoins et défis. Pour assurer le développement et la durabilité des universités européennes à venir, il est nécessaire de surveiller et d'évaluer l'impact lié à la vision à long terme de ces alliances.

Les alliances européennes pourraient être caractérisées en suivant le modèle proposé en 2005 par G Ahrne et N Brunsson \cite{ahrne_organizations_2005}, en appliquant le concept de meta-organisation pour désigner le regroupement de nombreuses organisations autour d'objectifs et projets commun. 

De nombreux framework et méthodologies ont été développés et proposés au fil des ans sur la théorie des changements et  d'évaluation de l'impact. Cependant, ces cadres peuvent ne pas fournir aux meta-organisations les outils et la flexibilité des organisations simples. Ceci prouve l'importance d'une méthodologie plus large pour les meta-organisations, tout en gardant une ancre pour chaque organisation partenaire.

Ce projet de recherche vise à développer un cadre pour les meta-organisation (dans ce contexte, les alliances d'universités européennes) pour surveiller et évaluer leur impact sur la société en utilisant des solutions orientées données.