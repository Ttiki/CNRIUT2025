Pour cette première année, nous avons établi une méthodologie pour récupérer les indicateurs et besoins de chaque tâche dans l'alliance UNITA. De ces interviews utilisant cette méthodologie orientée impact, nous avons pu récupérer une liste conséquente d'informations. Informations que nous avons validé avec chaque tâche, puis avec l'équipe de l'observatoire d'impact de UNITA (tâche) 5.4. Enfin, après un premier nettoyage des données, elles furent utilisées au sein du data warehouse de UNITA en tant que meta-data. Les choix et discussions n'étant pas encore finalisés, nous n'avons pas encore de résultat sur les indicateurs d'impact (plus difficile à demander et récolter). 
\begin{table}[h]
  \caption{Résultat des premières interviews}
  \begin{center}
    \begin{tabular}{|l|l|l|l|} \hline
    \textbf{Type d'indicateurs} & \textbf{Nombre d'indicateurs} \\ \hline
    Outputs & 160 \\ \hline
    Outcomes & 75 \\ \hline
    \end{tabular}
  \end{center}
\end{table}